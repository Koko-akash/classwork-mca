\documentclass[12pt]{article}

\usepackage{amsmath}
\usepackage{graphicx}
\usepackage{hyperref}
\usepackage{geometry}
\geometry{a4paper, margin=1in}

\title{\textbf{Mortal Kombat}}
\author{Akash Phukan \\ MCA 1st Year, Roll No: 05 \\ Dibrugarh University}
\date{\today}

\begin{document}

\maketitle

\begin{abstract}
Mortal Kombat (commonly abbreviated MK) is a popular series of fighting games created by Midway Games, which in turn spawned a number of related media. It is especially noted for its digitized sprites (which differentiated it from its contemporaries' hand-drawn sprites), and its mix of bloody and brutal action; its graphic Fatality killing moves contributed to the founding of the ESRB.
\end{abstract}

\section{Introduction}
The original Mortal Kombat was developed as a reaction to the popular Capcom fighting game Street Fighter II, with simpler controls and digitized graphics. The original idea of Mortal Kombat was thought up in 1989 along with storyline and game content, but not put to arcades until 1991 (and game systems). Some say the game's graphic violence was gratuitous, and was only included in order to generate a public outcry and controversy that would garner publicity for the game. Although highly controversial, the mix of realism and violence propelled Mortal Kombat to widespread and historic renown.

\section{Historical Background}
The original Mortal Kombat was developed as a reaction to the popular Capcom fighting game Street Fighter II, with simpler controls and digitized graphics. The original idea of Mortal Kombat was thought up in 1989 along with storyline and game content, but not put to arcades until 1991 (and game systems). Some say the game's graphic violence was gratuitous, and was only included in order to generate a public outcry and controversy that would garner publicity for the game. Although highly controversial, the mix of realism and violence propelled Mortal Kombat to widespread and historic renown.

\section{Characters}
Roster Characters:
\begin{itemize}
    \item \textbf{Johnny Cage}: "I mean look at this. Don't the rainbow colors on these mountains just scream Outworld?"
    Initially, Johnny Cage was a struggling Hollywood action movie star who sought to revitalize his career and reputation. He entered the Mortal Kombat tournament for the sole purpose of displaying his fighting skills and proving himself to skeptical film critics, as they believed Cage was nothing more than an actor who relied too much on stunts and camera work, and was not really a proficient fighter. Eventually, he would become one of the most important defenders of Earthrealm. Despite his materialism, airheadedness, and occasional immaturity, Cage is a brave and loyal warrior, although his antics frequently annoy his allies. For this reason, fans consider him to be the series' foremost comic relief character.
    
    \item \textbf{Liu Kang}: Liu Kang is one of the most popular and accessible characters in the series. He is one of Earthrealm's greatest warriors, having defeated many to prove his valor. Throughout the series, he has been gradually portrayed as the main hero, becoming champion after the first Mortal Kombat tournament. Liu Kang has found a nemesis in the traitorous Shang Tsung, a sorcerer who he has defeated on numerous occasions. He is often considered the greatest threat to the plans of Shang and his emperor, Shao Kahn. He is a member of the White Lotus Society and trains extensively at the Wu Shi Academy. It was at the Academy that he was murdered by a Deadly Alliance of Shang with Quan Chi.
    
    \item \textbf{Raiden}: Raiden is the eternal God of Thunder, protector of Earthrealm, and arguably one of the most powerful characters in all of Mortal Kombat. After the second defeat of Shinnok, he ascended to the status of Elder God. As a result of his divine nature, he possesses many supernatural abilities, such as the ability to teleport, control lightning, and fly. He is also more prone to thinking in terms of eternity rather than normal human lifespans, and has a radically different outlook on life.
\end{itemize}

\section{Story}
500 years ago, the annual Shaolin Tournament, long the most prestigious fighting tournament in the world, was interrupted by the appearance of an old sorcerer and a strange four-armed creature, who entered the tournament and defeated the Great Kung Lao. This Shokan warrior was the half-human, half-dragon fighter named Goro, and he became the ultimate fighting champion for the next five hundred years. This was all part of Shang Tsung's plan to tip the balance into chaos and help Outworld conquer the Earth Realm.
\begin{itemize}
    \item \textbf{First Mortal Kombat game}: The first Mortal Kombat game takes place in Earthrealm (Earth) where seven different warriors with their own reasons for entering the tournament with the prize being the continued freedom of their realm under threat of a takeover by Outworld. Among the established warriors were Liu Kang, Johnny Cage, and Sonya Blade. With the help of the thunder god Raiden, the Earthrealm warriors were victorious, and Liu Kang became the new champion of Mortal Kombat.[19] In Mortal Kombat II, unable to deal with his minion Shang Tsung's failure, Outworld Emperor Shao Kahn lures the Earthrealm warriors to Outworld for a do-over, winner-take-all tournament, where Liu Kang eventually defeats Shao Kahn. By the time of Mortal Kombat 3, Shao Kahn merged Edenia with his empire and revived its former queen Sindel in Earthrealm, combining it with Outworld as well. He attempts to invade Earthrealm, but is ultimately defeated by Liu Kang once more. After the Kahn's defeat, Edenia was freed from his grasp and returned to a peaceful realm, ruled by Princess Kitana. The following game, Mortal Kombat 4, features the fallen elder god Shinnok attempting to conquer the realms and kill Raiden. He is defeated by Liu Kang.
    
    \item \textbf{Mortal Kombat}: In the 2011 Mortal Kombat soft reboot, the battle of Armageddon culminated in only two survivors: Shao Kahn and Raiden. On the verge of death by the former's hand, the latter sent visions to his past self in a last-ditch attempt to prevent this outcome. Upon receiving the visions, the past Raiden attempts to alter the timeline to avert Armageddon amidst the tenth Mortal Kombat tournament, during the original game. His attempts to alter history mean that events play out differently to the original series. While he succeeds in preventing Shao Kahn's victory with help from the Elder Gods, he accidentally kills Liu Kang in self-defense and loses most of his allies to Queen Sindel, leaving Earthrealm vulnerable to Shinnok and Quan Chi's machinations.
    
    \item \textbf{Mortal Kombat X }: Mortal Kombat X sees Shinnok and Quan Chi enacting their plan, leading an army of undead revenants of those that were killed in Shao Kahn's invasion of Earthrealm. A team of warriors led by Raiden, Johnny Cage, Kenshi Takahashi, and Sonya Blade oppose them, and in the ensuing battle, Shinnok is imprisoned within his amulet and various warriors are resurrected and freed from his control, though Quan Chi escapes. Twenty-five years later, the sorcerer resurfaces alongside the insectoid D'Vorah to facilitate Shinnok's return. A vengeful Scorpion kills Quan Chi, but fails to stop him from freeing Shinnok. To combat him, Cassie Cage, daughter of Johnny Cage and Sonya Blade, leads a team composed of the next generation of Earthrealm's heroes in defeating him. With Shinnok and Quan Chi defeated, Liu Kang and Kitana's revenants assume control of the Netherrealm while Raiden taps into Shinnok's amulet.
    
    \item \textbf{Mortal Kombat 11 }: Mortal Kombat 11 and its expansion, Aftermath, sees the architect of time and Shinnok's mother, Kronika, working to alter the timeline following her son's defeat and Raiden's tampering with her work. In doing so, she brings past versions of the realm's heroes to the present, aligning herself with some while the rest work to defeat her. After nearly killing Liu Kang a second time, Raiden discovers Kronika has manipulated them into fighting across multiple timelines as she fears their combined power. Despite her interference and attacks by her minions, Raiden gives Liu Kang his power, turning him into a god of fire and thunder so he can defeat Kronika. In the Aftermath expansion, it is revealed that Liu Kang inadvertently destroyed Kronika's crown, the item needed to restart the timeline. Her defeat also revives Shang Tsung, who was absent in the base game due to his imprisonment by Kronika. To recover the crown, Liu Kang sends Shang Tsung and other Earthrealm heroes back in time to obtain it before Kronika, though Shang Tsung manipulates events so that he comes into possession of the crown. At the end, either Liu Kang or Shang Tsung becomes the Keeper of Time, depending on the player's choice (who they want to fight with in the final battle) and the outcome of the battle.
\end{itemize}

\section{Legacy and cultural impact}
According to IGN, during the 1990s "waves of imitators began to flood the market, filling arcades with a sea of blood from games like Time Killers, Survival Arts, and Guardians of the Hood. Mortal Kombat had ushered in an era of exploitation games, both on consoles and in arcades, all engaging in a battle to see who can cram the most blood and guts onto a low-res screen."[1] Notable Mortal Kombat clones, featuring violent finishing moves and/or digitized sprites, included Bio F.R.E.A.K.S., BloodStorm, Cardinal Syn, Catfight, Eternal Champions, Kasumi Ninja, Killer Instinct, Mace: The Dark Age, Primal Rage, Street Fighter: The Movie, Tattoo Assassins, Thrill Kill, Ultra Vortek, Way of the Warrior, and Midway's own War Gods.[193][194] John Tobias commented: "Some of the copycat products back then kind of came and went because, on the surface level, the violence will attract some attention, but if there's not much to the product behind it, you're not going to last very long."[

\section{Conclusion}
The series has a reputation for high levels of graphic violence, including, most notably, its fatalities, which are finishing moves that kill defeated opponents instead of knocking them out. Controversies surrounding Mortal Kombat, in part, led to the creation of the Entertainment Software Rating Board (ESRB) video game rating system. Early games in the series were noted for their realistic digitized sprites and an extensive use of palette swapping to create new characters. Following Midway's bankruptcy, the Mortal Kombat development team was acquired by Warner Bros. Entertainment and re-established as NetherRealm Studios.

\end{document}
