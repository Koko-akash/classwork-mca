\documentclass[12pt]{article}

\usepackage{amsmath}
\usepackage{graphicx}
\usepackage{hyperref}
\usepackage{geometry}
\geometry{a4paper, margin=1in}

\title{\textbf{The Boys}}
\author{Akash Phukan \\ MCA 1st Year, Roll No: 05 \\ Dibrugarh University}
\date{\today}

\begin{document}

\maketitle
\newpage
\begin{abstract}
\begin{figure}
    \centering
    \includegraphics[width=0.5\linewidth]{images/homelanderLatex.jpg}
    \caption{"You guys are the real heroes."}
    \label{fig:enter-label}
\end{figure}
\vfill
\centering \textbf{\Large Never meet your heroes}
\end{abstract}
\newpage
\section{Introduction}
The Boys is an irreverent take on the superhero genre. It explores what happens when superheroes abuse their powers instead of using them for good. These mighty beings — who are popular as celebrities, as influential as politicians, and as revered as gods — find themselves hunted by a non-powered group of vigilantes.

A CIA black-ops team, their job is to police, monitor and potentially stop The Seven, a group of vainglorious superheroes. They were founded by Billy Butcher, their current leader.

This band of brothers is out to expose the corruption they believe the Seven represent. Although they may not have superpowers, they do have a lot of "blue-collar grit and [a] willingness to fight dirty."



\section{Historical Background}
The Boys premiered its first season of eight episodes on July 26, 2019. A second season premiered on September 4, 2020, with the third season following on June 3, 2022. In June 2022, the series was renewed for a fourth season, which premiered on June 13, 2024. In May 2024, the series was renewed for a fifth and final season, which is expected to premiere in 2026. As part of a shared universe, a spin-off web series (Seven on 7) premiered on July 7, 2021, an adult animated anthology series (Diabolical) premiered on March 4, 2022, and a second live-action television series (Gen V) premiered on September 29, 2023.
\newpage
\section{Characters}
\begin{itemize}

    \item \flushleft \textbf{Homelander}:  \\
    \centering
    \includegraphics[width=0.2\linewidth]{images/homelandah.jpg} \\
    \flushleft
     Homelander is the leader of The Seven, the strongest Supe in the world, and the archenemy of Billy Butcher and The Boys. With the face of a movie star and the powers of a god, Homelander is considered the greatest superhero alive. Not only can he fly, but he possesses super strength and super durability far beyond the capacity of other superheroes, super senses (sight, hearing, etc.), X-ray vision and laser vision.

    On the surface, he's affable, modest, and sincere; the ultimate boy scout, an American treasure, a God-loving patriot. But just like regular mortals, even superheroes have secrets.
\newpage
    \item \flushleft \textbf{Billy Butcher}:  \\
    \centering
    \includegraphics[width=0.2\linewidth]{images/billy butcher.jpg} \\
    \flushleft
    He is the leader of the eponymous team of vigilantes who are bent on taking down Vought and the Seven by whatever means necessary. A former member of the British special forces turned vigilante; Billy Butcher is as charming as he is cunning. He's a force of nature, who can talk almost anyone into anything, either through a smile or brute force – or sometimes both. He's consumed by one mission in life: to destroy superheroes. But this personal vendetta is driven by his hatred for one Supe in particular: Homelander. Butcher is determined to get revenge on Homelander, no matter the cost, and he won't let anyone, or anything stand in his way.
\newpage
    \item \flushleft \textbf{Soldier Boy}:  \\
    \centering
    \includegraphics[width=0.2\linewidth]{images/soldier-boy.jpg} \\
    \flushleft
    Benjamin, better known as Soldier Boy, is one of the two overarching antagonists (alongside Stan Edgar) of the Amazon series The Boys, where he served as one of two main antagonists (alongside Homelander) of Season Three and the overarching antagonist of Season Four.

    He was America's first and greatest superhero before Homelander and the former leader of the superhero team Payback. As a young man, Soldier Boy helped good triumph over evil in World War II. However, it is not made clear how much of that is true, as The Legend claimed that his participation in the war was Vought propaganda whereas Stan Edgar actually claims the opposite by stating that he killed Germans by the dozens. With his superhero team Payback by his side, he was said to have fought for liberty and justice for all until his disappearance during a botched military operation in Nicaragua, with the cover story being that he heroically sacrificed his own life to save America from a nuclear power plant meltdown in 1984. In reality, Soldier Boy was betrayed by his team for his aggressive behavior towards them and, with Vought's approval, was sold out to the Russians, who would later conduct dozens of agonizing experiments on him over the next 3 decades. Much like Homelander, Soldier Boy was esteemed by the public but due to the news, Starlight's recent exposé about his recent actions and Vought's slanderous lies, his fame and reputation crumbled.
\end{itemize}

\section{Development}
Between 2008 and 2016, a film adaptation of The Boys had been in various stages of development at both Columbia Pictures and Paramount Pictures.

On April 6, 2016, it was announced that Cinemax was developing a television series adaption of the comic book. The production was being developed by Eric Kripke, Evan Goldberg, and Seth Rogen. Kripke was set to write the series while Goldberg and Rogen were set to direct. Executive producers were reported to include Kripke, Goldberg, Rogen, Neal H. Moritz, Pavun Shetty, Ori Marmur, James Weaver, Ken Levin, and Jason Netter. Garth Ennis and Darick Robertson were set as co-executive producers. Production companies involved with the series included Point Grey Pictures, Original Film, and Sony Pictures Television.

On November 8, 2017, it was announced that Amazon had given the production a series order for a first season consisting of eight episodes. The series had reportedly been in development at Amazon for a number of months preceding the series order announcement. It was also reported that the previously announced creative team was still attached to the series.

Kripke wanted to retain a sense of reality to the show, and to keep the writers disciplined decided "Anything that comes out of this drug is viable, and anything that doesn't we're not allowed to do". He did not want to fall into the overused convention of killing off female characters to motivate the heroes and also saw an opportunity to surprise readers of the comics by changing the story of Butcher's wife Becky.

On April 30, 2018, it was announced that Dan Trachtenberg would direct the series' first episode. He replaces Rogen and Goldberg who dropped out due to scheduling conflicts.

Ahead of the series premiere, on July 19, 2019, it was announced that Amazon had renewed the series for a second season. The eight scripts for the second season were completed by November 2019.

Ahead of the second season premiere, on July 23, 2020, Amazon renewed the series for a third season at the aftershow hosted by Aisha Tyler for San Diego Comic-Con@Home.

\section{Reception}
On Rotten Tomatoes, the first season holds an approval rating of 84\% based on 98 reviews, with an average rating of 7.72/10. The website's critical consensus reads, "Though viewer's mileage may vary, The Boys violent delights and willingness to engage in heavy, relevant themes are sure to please those looking for a new group of antiheroes to root for." On Metacritic, it has a weighted average score of 74 out of 100, based on reviews from 19 critics, indicating "generally favorable reviews".

\end{document}
