\documentclass[13pt]{beamer}
\usetheme{Boadilla}
\usetheme{CambridgeUS}
\usecolortheme{dolphin}
\usepackage{graphicx}
\usepackage{xcolor}

% set colors
\definecolor{myNewColorA}{RGB}{126,12,110}
\definecolor{myNewColorB}{RGB}{165,85,154}
\definecolor{myNewColorC}{RGB}{203,158,197}
\setbeamercolor*{palette primary}{bg=myNewColorC}
\setbeamercolor*{palette secondary}{bg=myNewColorB, fg = white}
\setbeamercolor*{palette tertiary}{bg=myNewColorA, fg = white}
\setbeamercolor*{titlelike}{fg=myNewColorA}
\setbeamercolor*{title}{bg=myNewColorA, fg = white}
\setbeamercolor*{item}{fg=myNewColorA}
\setbeamercolor*{caption name}{fg=myNewColorA}
\usefonttheme{professionalfonts}
\usepackage{natbib}
\usepackage{hyperref}
%

\title{Structures}
\author{Akash Phukan}
\institute{Roll No: 05, MCA 1st SEM}
\date{07-11-2024}

\begin{document}
\maketitle

\begin{frame}{Contents}
\begin{enumerate}
    \item What are structures?
    \item Why do we need them?
    \item Syntax:
    \begin{itemize}
        \item Structure Declaration
        \item Structure Definition
        \item Access Structure Members
    \end{itemize}
    \item Real World Example.
    \item Example in code.
    \item Conclusion.
\end{enumerate}
\end{frame}

\begin{frame}{What is a Structure?}
\begin{itemize}
    \item The structure in C is a user-defined data type that can be used to group heterogeneous items (of possibly different types) into a single type. 
    
    \item The struct keyword is used to define the structure in the C programming language.  
    
    \item The items in the structure are called its member and they can be of any valid data type.(Such as, int, char, float etc)

    \item C allows Nesting of structures. They can be done in two ways:
    \begin{itemize}
        \item By separate nested structure. (creating instances)
        \item By embedded nested structure. (structure in other structure)
    \end{itemize}
\end{itemize}
\end{frame}

\begin{frame}{Why do we need structure?}
\begin{itemize}
    \item Structure is a user-defined data in C language will allows us to combine data of different types together. 
    \item Structure helps to construct a complex data type which is more meaningful.
    \item C structures can be used to store huge data.
\end{itemize}
\end{frame}

\begin{frame}{Syntax}
{\textbf{Structure Declaration}}
\begin{itemize}
    \item We have to declare structure in C before using it in our program. 
    \item In structure declaration, we specify its member variables along with their datatype.
    \item We can use the struct keyword to declare the structure in C using the following syntax:
\end{itemize}
\begin{block}{Syntax:}
    struct structureName \{ \\
    dataType memberName1; \\
    dataType memberName1; \\
    .... \\
\};
\end{block}
\end{frame}

\begin{frame}{Syntax}
{\textbf{Structure Definition}}
\begin{itemize}
    \item To use structure in our program, we have to define its instance. We can do that by creating variables of the structure type.
    \item We can define structure variables using two methods.
\end{itemize}

\end{frame}

\begin{frame}{Syntax}
{\textbf{Structure Definition}}
\begin{block}{1. Structure Variable Declaration with Structure Template}
    struct StructureName \{ \\
    dataType memberName1; \\
    dataType memberName1; \\
    .... \\
    .... \\
\}variable1, varaible2, ...;
\end{block} 
\begin{block}{2. Structure Variable Declaration after Structure Template}
    // structure declared beforehand \\
struct structureName variable1, variable2, .......;
\end{block}
\end{frame}

\begin{frame}{Syntax}
{\textbf{Access Structure Members}} 
\begin{itemize}
    \item We can access structure members by using the ( . ) dot operator.
\end{itemize}
\begin{block}{Syntax}
    structureName.member1; \\
    strcutureName.member2;
\end{block}
\end{frame}

\begin{frame}{Real World Example}
\begin{itemize}
    \item For this example, let us consider a Garage.
    \item In the garage we want to store 3 cars(say).
\end{itemize}
\begin{figure}
    \centering
    \includegraphics[width=0.8\linewidth]{presentation/example.png}
    \caption{Different cars that we want to store in the garage}
    \label{fig:enter-label}
\end{figure}
\end{frame}

\begin{frame}
    Each car will be an entity in the garage. And each car will have different  characteristics. \\
    To better understand this, take a look at the following figures: \\
    \begin{figure}
        \centering
        \includegraphics[width=0.8\linewidth]{presentation/car.png}
        \caption{A car and its characteristics}
        \label{fig:1}
    \end{figure}
\end{frame}

\begin{frame}
    \begin{figure}
        \centering
        \includegraphics[width=0.8\linewidth]{presentation/garage.png}
        \caption{A Garage containing Cars}
        \label{fig:2}
    \end{figure}
\end{frame}

\begin{frame}{Example in code}
    \begin{itemize}
        \item Let us see how the above real life example would look like when translated it into code.
    \end{itemize}
\end{frame}

\begin{frame}
    Here we are creating a Structure Car, which will have heterogeneous data members. \\
    In other words Structure Declaration.
    \begin{figure}
        \centering
        \includegraphics[width=0.8\linewidth]{presentation/structure car.png}
        \caption{Structure for a Car}
        \label{fig:3}
    \end{figure}
\end{frame}

\begin{frame}
    In the main function, we will be creating an array variable for our use-defined data structure Car. Through the use of this array variable we can access different members of the structure. \\
    In computing terms, its called Structure Definition and Access Structure Members.
\end{frame}

\begin{frame}
    \begin{figure}
        \centering
        \includegraphics[width=1.0\linewidth]{presentation/main fn.png}
        \caption{Main Function}
        \label{fig:4}
    \end{figure}
\end{frame}

\begin{frame}
    \begin{center}
    \begin{large}
    \textcolor{magenta}{\textbf{Then we compile and run our program...}}
    \end{large}
    \end{center}
\end{frame}

\begin{frame}
    Output: \\
    When we run the code we will get this output. \\
    Then we enter our data. \\
    \begin{figure}
        \centering
        \includegraphics[width=0.5\linewidth]{presentation/op input.png}
        \caption{Output}
        \label{fig:5}
    \end{figure}
\end{frame}

\begin{frame}
    After we have entered our data, we click enter. \\
    And we get this output. \\
    \begin{figure}
        \centering
        \includegraphics[width=0.6\linewidth]{presentation/op output.png}
        \caption{Output}
        \label{fig:6}
    \end{figure}
\end{frame}

\begin{frame}
    \begin{itemize}
        \item From the above example, we can say that Structures makes it way easier to manage big data of different types.
        \item The code for the management system was also readable and not messy. 
    \end{itemize}
\end{frame}

\begin{frame}{Conclusion}
    \begin{itemize}
        \item Structures help organize data faster, which can increase productivity.
        \item Structures make code easier to read.
        \item Structures can store data of different types, such as int, float, and char, under one name.
    \end{itemize}
\end{frame}

\begin{frame}
    \begin{center}
    \begin{large}
    \textcolor{magenta}{\textbf{Thank You!}}
    \end{large}
    \end{center}
\end{frame}
\end{document}
